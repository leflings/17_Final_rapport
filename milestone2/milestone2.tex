\documentclass[a4paper]{article}
\usepackage[utf8]{inputenc}
\usepackage[danish]{babel}
\usepackage[T1]{fontenc}
\renewcommand{\danishhyphenmins}{22}
\renewcommand{\arraystretch}{1.3}
\usepackage{amsmath,amssymb,bm,mathtools,rotating}

\usepackage{float}
\usepackage{listings,color}

\usepackage[top=3cm,bottom=3cm,left=3cm,right=3cm]{geometry}


\title{02324: Milestone 2}
\author{
}

\begin{document}

\tableofcontents

\vspace{5cm}

\section{Timeregnskab} % (fold)
\label{sec:Timeregnskab}
\begin{tabular}{l l | c c c c c | c}

  \emph{Studienr} & \emph{Navn}
  & \begin{sideways}Design\end{sideways} 
  & \begin{sideways}Implementation\end{sideways} 
  & \begin{sideways}Test\end{sideways} 
  & \begin{sideways}Dokumentation\end{sideways} 
  & \begin{sideways}Andet\end{sideways} 
  & \begin{sideways}Total\end{sideways} \\
  \hline
  s113933 & Flemming Madsen              & 5 & 13 & 8  & 7 & 4  & 37 \\
  s093018 & Christian Walldeskog Nielsen & 7 & 14 & 10 & 6 & 0  & 37 \\
  s103799 & Martin Roland Hartvig        & 0 & 24 & 12 & 10 & 0 & 46 \\
  s113618 & Tobias Brasch                & 2 & 37 & 4  & 3 & 3  & 49 \\
  s113610 & Lavdrim Elmazi               & 0 & 0  & 5  & 0 & 15 & 20 \\
  \hline
  -       & -                            & 14 & 88 & 39 & 26 & 22 & 189
  
\end{tabular}

% section Timeregnskab (end)

\clearpage

\section{Vejledning} % (fold)
\label{sec:Vejledning}

\subsection*{WebSite} % (fold)
\label{sub:WebSite}

Vores hjemmeside forefindes i source-folderen \texttt{WebSite}. Under pakken \texttt{web.controller} ligger filen \texttt{WebController.java}. Køres denne startes hjemmesiden (givet at en TomCat server allerede er konfigureret)
\\
\\
Den primære bruger på hjemmesiden er:
\begin{itemize}
  \item \textbf{ID:} 10
  \item \textbf{PW:} 02324
\end{itemize}

% subsection WebSite (end)

\subsection*{Vægtsimulator} % (fold)

Vores vægtsimulator (med GUI) findes under source-folderen \texttt{Simulator}. Fra pakken \texttt{simulator} køres filen \texttt{Scale.java} og simulatoren starter. Denne lytter som standard på port 8000.

% subsection Vægtsimulator (end)

\subsection*{Afvejnings Styrings Enhed} % (fold)
\label{sub:Afvejnings Styrings Enhed}

Vores afvejningsstyringsenhed findes i source-folderen \texttt{ASE}. Fra pakken \texttt{wcu} køres \texttt{WCU.java}. Ved opstart bliver man bedt om at indtaste ip-adresse på den vægt man vil forbinde til. Hvis ASE'en skal køres op imod simulatoren, er adressen \texttt{localhost}

% subsection Afvejnings Styrings Enhed (end)

% section Vejledning (end)

\clearpage

\section{Poster} % (fold)

Ingen poster

% section Poster (end)

\clearpage

\section{Milestone 2} % (fold)

\subsection{Projektets status} % (fold)

I forhold til sidste milestone's tidsplan er projektet forløbet planmæssigt. Programmeringsdelen er afsluttet, og programmerne er rent funktionelt og design mæssigt hvor vi ønsker dem.

% subsection Projektets status (end)

\subsection{Vægtsimulator} % (fold)

Vægtsimulatoren implementerer alle de funktioner der er nødvendige for at vores ASE kan gennemføre afvejning af komponenter til produktbatches.
\begin{itemize}
  \item S
  \item T
  \item Z
  \item RM20 8
  \item RM30
  \item RM39
  \item RM49
  \item P110
  \item P111
\end{itemize}

% subsection Vægtsimulator (end)

\subsection{Afvejnings Styrings Enhed} % (fold)
\label{sub:Afvejnings Styrings Enhed}

Vores ASE er istand at gennemføre afmålinger af komponenter til produktbatches. Det er ikke nødvendigt at afslutte produktionen af et batch, man kan nøjes med at afmåle enkelte komponenter og senere genoptage arbejdet. Komponenterne bliver løbende gemt som de bliver afvejet. Endvidere er det ikke muligt at arbejde på et allerede afsluttet produktbatch.

Der er lagt vægt på at assistere operatøren bedst muligt under arbejdet. Således oplyses operatøren blandt andet om hvilke råvare-batches der er tilgængelige for et givet komponent (der tages højde for at batchet har en tilstrækkelig mængde tilbage osv.), i hvilket område afmålingen skal være (netto +- tolerance).

% subsection Afvejnings Styrings Enhed (end)

\subsection{Website} % (fold)

Hjemmesiden indeholder fuld funktionalitet i forhold til Use Case 1-5 i opgaveformuleringen.

Der er benyttet et gennemgående farvetema på hjemmesiden, således at blå knapper er til oprettelse af data, hvide knapper er til redigering/visning af data og røde knapper er til sletning af data.

Alle input valideres både client-side og server-side. Client-side validering giver mulighed for at give brugeren visuel feedback så snart værdier indtastes, således at det ikke er nødvendigt at submit'e data før man får feedback.

Der er desuden udviklet en mobil version af hjemmesiden, der specielt er tiltænkt at kunne benyttes af Operatører, sådan at Værkføreren ikke længere behøver at udskrive en produktionsforskrift, men operatøren istedet kan tilgå en speciel designet hjemmeside der giver alle nødvendige informationer til fremstilling af et produkt batch

% subsection Website (end)

\subsection{Tidsplan} % (fold)

Indtil projektets afslutning er fokus primært på rapportskrivning, samt tilpasning af detaljer og refaktorering af kode.

% subsection Tidsplan (end)

% section Milestone 1 (end)


\end{document}


